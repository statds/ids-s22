\documentclass{article}
\usepackage[hmargin={0.5in, 0.5in}, vmargin={0.8in, 0.8in}]{geometry}
\usepackage{booktabs}
\usepackage[table]{xcolor}
\usepackage{hyperref}
\usepackage[anythingbreaks]{breakurl}
\usepackage{multicol}

\usepackage[compact]{titlesec}

\newcommand{\pkg}[1]{{\normalfont\fontseries{b}\selectfont #1}}
\let\proglang=\textsf
\let\code=\texttt

\usepackage{color}
\newcommand{\jy}[1]{\textcolor{red}{Updated: (#1)}}

\usepackage{enumitem}
\setlist{parsep=0pt, leftmargin=4mm, topsep=0pt, itemsep=8pt}
% \setlist{nolistsep}

\usepackage{fancyhdr}
\pagestyle{fancy}
\renewcommand{\headrulewidth}{0.4pt}
\renewcommand{\footrulewidth}{0.4pt}


\lhead{\sf STAT 5255/3255}
\rhead{Spring 2022}
\lfoot{\sf jun.yan@uconn.edu}
\rfoot{\url{http://www.stat.uconn.edu/~jyan/}}

\begin{document}
% \maketitle

\title{STAT 5255/3255: Introduction to Data Science}
\date{January 18, 2022}

\maketitle

\thispagestyle{fancy}

\begin{multicols}{2}
\begin{description}
\item[Instructor:]
  Jun Yan\\
  % Austin 328\\
  % 860/486-3416\\
  jun.yan@uconn.edu

\item[Lectures:] 
  TuTh 3:30--4:45 pm\\
  FSB 202 (if in-person)\\
  WebEx (if online):  
  Log into HuskyCT; Go into the course; Click on WebEx on the left menu and
  then you will see the lecture meetings. Recordings will be here too when
  they are available.

\item[Office Hours:] 
  W 12:30 -- 1:30 pm @ WebEx virtual office
  \url{https://uconn-cmr.webex.com/meet/juy07002}
  or by appointment.

item[Grader:] Yifan Li (\href{mailto:
   yifan.3.li@uconn.edu}{yifan.3.li@uconn.edu})
  
\item[Prerequisites:]
  STAT 5255: Open to graduate students in Statistics, others with
  permission. Not open for credit to students who have passed STAT
  3255.

  STAT 3255: STAT 2255 and STAT 3115Q, or instructor consent.
    
\item[Course Description:]
  Introduction to data science for effectively storing, processing,
  visualizing, analyzing and making inferences from data to enable
  decision making. Topics include project management, data
  preparation, data visualization, statistical modeling, machine
  learning, distributed computing and ethics.
  
\item[Course Materials:]
  Lecture notes, assignments, sample code, datasets, and other course
  material will be posted on the HuskyCT course website (available at
  \url{https://lms.uconn.edu/}).
  
  Please visit this site often to ensure timely obtainment of materials.
  
%{\red The lecture notes will be available online before each class.}
\item[Recommended Textbooks:]\hspace{0pt}
  \begin{enumerate}
  \item 
    ``Python Data Science Handbook: Essential Tools for Working with
    Data,'' First Edition, by Jake VanderPlas, O’Reilly Media, 2016.
 
  \item
    ``Python for Data Analysis: Data Wrangling with Pandas, NumPy, and
    IPython.''  Second Edition by Wes McKinney, O’Reilly Media, 2018.
  \item
    \href{https://www.practicaldatascience.org/html/not_a_mids_student.html}{Practical
    Data Science at Duke}.
  \end{enumerate}

\item[Computing:]
  \proglang{Python} will be the primary computing environment. You are
  free to choose any interactive development environment (Jupyter
  Notebook, VS Code, Emacs, etc.). Checkout
  \href{https://google.github.io/styleguide/pyguide.html}{Google
  Python Style Guide} and practice the recommendations.

  Git will be used for project management and work flows, and GitHub
  Classroom will be used for homework submission and grading. Learn
  and practice \href{https://gitexercises.fracz.com}{online}.

  Command line operation is essential. Point-and-clicks are enemies or
  replicability and automation. Pick the basics from any tutorial
  (e.g.,
  \href{https://ubuntu.com/tutorials/command-line-for-beginners}{Ubuntu}).

  Always be ready to learn new things.

\item[Grading:]
The grade for this course will be based on:
\begin{center}
  \rowcolors{2}{blue!20}{gray!20}
  \begin{tabular}{lrr}
    \toprule
    Category                & 3255   & 5255\\
    \midrule
    Style                       &  5\%   & 5\%\\
    Participation           &  5\%   & 5\%\\
    Homework              &  10\% & 10\%\\
    Topic Presentations&  20\% & 20\%\\
    Midterm Project       &  20\% & 20\%\\
    Final Presentation    &  30\% & 0\%\\
    Final Report             &  0\% & 30\%\\
    \bottomrule
  \end{tabular}
\end{center} 

\item[Style:]
  If the grader or instructor has to reread a solution several times
  to find the train of thought, or if a solution was illegible,
  ambiguous, or incoherent, or if explicitly stated styles are
  violated, one point of style is deducted at each occurrence until
  the 5 points are exhausted.

\item[Participation:]  Participation in the remote setting includes
  active use of HuskyCT Discussion Boards, answering questions posed
  during lecture, and evaluation of presentations (required) from the
  undergraduate students.

\item[Homework:]\hspace{0pt}
\begin{itemize}
\item Homeworks will be assigned roughly weekly throughout the
  semester.  Students may consult amongst themselves or with the
  instructor, but each student must submit his/her own work.
  
\item All completed assignments are to be submitted by the due date
  unless extra flexibility is allowed.. 
Assignments will be accepted up to 48 hours late, but with penalty.  If the
submission is within 48 hours of the due date and time, total amount of credit
available will be a linear function of time-beyond-due, ranging from 95\%--50\%
of the total points. Submissions over 48 hours late will not be graded and will
receive no credit.
		
\item No credit will be given for submitted assignments exhibiting
  duplication or copying of solutions (from peers or existing
  solutions).
\end{itemize}

\item[Topic Presentations:]
  Students pick up presentation topics from a task board set up by the
  class, prepare for it, and present to the class. All the materials will be part
  of a Git repository for a class-note co-developed by all class members.

\item[Midterm Project:]
  The midterm project will be an assigned one. It requires completion
  of the analysis component of a full data science project
  and presenting results in a talk (undergraduates) or report (graduates).
  
\item[Final Project:]
  The topic will be of your choice. It should show
  the whole cycle of a real data science project. Consider projects at
  data science competition sites (e.g., \url{https://kaggle.com},
  \url{https://community.amstat.org/dataexpo/home}).
  A proposal is due 11:59 pm, Friday, April 1.

  Undergraduates in 3255 will give 15-minutes presentations in an
  order to be randomly determined by the class.
  
  Graduate students in 5255 will submit a final written report by
  11:59pm, Friday, May 6, 2022.

\end{description}
\end{multicols}

\section*{COVID Guidelines}
  See \url{https://covid.uconn.edu/campus-guidelines}.

  The classes may become in-person as university policies allow.

\section*{University Policies and Academic Integrity}
  University Policies and Academic Integrity
This course adheres to the policies from the University Senate, the
Office of Institutional Equity, the Office of the Provost, Community
Standards, and the Graduate School. See
\url{http://provost.uconn.edu/syllabi-references}
for more infor- mation. In particular, the policy on Academic
Integrity state:
\begin{quote}
A fundamental tenet of all educational institutions is academic
honesty; academic work depends upon respect for and acknowledgement of
the research and ideas of others. Mis- representing someone else’s
work as one’s own is a serious offense in any academic setting and it
will not be condoned. Academic misconduct includes, but is not limited
to, providing or receiving assistance in a manner not authorized by
the instructor in the creation of work to be submitted for academic
evaluation (e.g. papers, projects, and examinations); any attempt to
influence improperly (e.g. bribery, threats) any member of the
faculty, staff, or administration of the University in any matter
pertaining to academics or research; presenting, as one’s own,the
ideas or words of another for academic evaluation; doing
unauthorized academic work for which another person will receive
credit or be evaluated; and presenting the same or substantially the
same papers or projects in two or more courses without the explicit
permission of the instructors involved. A student who knowingly
assists another student in committing an act of academic misconduct
shall be equally accountable for the violation, and shall be subject
to the sanctions and other remedies described in The Student Code.
\end{quote}

\section*{Recording Lectures}
On-line classes will be conducted over WebEx Meetings. The online
lectures will be recorded by default and available automatically
afterwards. The recording feature for others in attendance will be
disabled so that no one else will be able to record a session. In
order to protect student privacy and intellectual property rights,
students are prohibited from recording any session, or any portion of
a session, by other means. The sharing of any
recorded content without my written permission is prohibited. If you
would like to ensure your likeness is not captured during an online
class, please turn your camera off. For recordings conducted in
person, please alert me to any concerns so that I may take steps to
help ensure you are not recorded.

Please remember that unauthorized recording or sharing of course
content may be considered a violation the law, University policy,
and/or The Student Code.

\section*{Disclaimer}
The professor reserves the right to make changes to the syllabus as
necessitated by circumstances.


% \label{LastPage}
\end{document}
